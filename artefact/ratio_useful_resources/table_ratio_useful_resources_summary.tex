\renewcommand{\arraystretch}{1.1}
\begin{table}[htbp]
	\begin{center}
		\resizebox{\textwidth}{!}{\begin{tabular}{|c|c|c|c|c|c|c|c|c|c|c|}
			\hline
            Approach/ Query Template & D1 & D2 & D3 & D4 & D5 & D6 & D7 & S1 & S4 & S5 \\
			\hline
            shape index (\%) & $\boldsymbol{14^{42}_{6}}$ & $\boldsymbol{38^{91}_{11}}$ & $\boldsymbol{52^{62}_{37}}$ & $\boldsymbol{15^{20}_{3}}$ & $\boldsymbol{9^{13}_{2}}$ & $18^{41}_{7}$ & $8^{11}_{2}$ & $\boldsymbol{12^{12}_{12}}$ & $6^{6}_{6}$ & $\boldsymbol{10^{11}_{4}}$ \\
            \hline
			type index (\%) & $4^{6}_{4}$ & $35^{86}_{8}$ & $51^{61}_{36}$ & $9^{15}_{3}$ & $8^{12}_{2}$ & $\boldsymbol{20^{42}_{8}}$ & $\boldsymbol{9^{13}_{2}}$ & $4^{8}_{1}$ & $\boldsymbol{80^{100}_{50}}$ & $3^{4}_{2}$ \\
			\hline
		\end{tabular}}
	\end{center}
	\caption{
        For most queries, the percentage of query-relevant resources is low (cell values are shown as $\text{avg}^{\text{max}}_{\text{min}}$). 
		shape index generally matches or outperforms the type index, except for templates D6, D7, and S4: D6 and D7 perform ~1\% worse, while S4 reaches 100\% with type index versus only 6\% with the shape index.
		}
	\label{tab:ratioUsefulResources}
\end{table}