\begin{table}
	\begin{center}
		\begin{tabular}{|l|l|l|l|}
			\hline
			query template & relation execution time & p-value & average ratio HTTP request \\
			\hline
			D1 & lesser & 1.14E-36 & 0.57 \\
			\hline
			D2 & lesser & 4.42E-04 & 0.88 \\
			\hline
			D3 & similar & 7.47E-01 (RH) & 0.97 \\
			\hline
			D4 & lesser & 2.07E-17 & 0.65 \\
			\hline
			D5 & lesser & 5.58E-03 & 0.88 \\
			\hline
			D6 & similar & 2.56E-01 (RH) & 1.12 \\
			\hline
			D7 & similar & 7.83E-01 (RH) & 1.12 \\
			\hline
			S1 & lesser & 1.12E-83 & 0.33 \\
			\hline
			S4 & greater & 3.76E-22 & 13.00 \\
			\hline
			S5 & lesser & 3.12E-17 & 0.44 \\
			\hline
		\end{tabular}
	\end{center}
	\caption{Table comparing the shape index approach to the state-of-the-art. RH, indicate that the p-value is associated to the rejected hypothesis. Every query performs better or similarly to the state-of-the-art with the shape index approach except for interactive-short-4. One might expect the average ratio of HTTP requests for D6 and D7 to be one. 
In our implementation, however, the query subsumption algorithm naively retrieved nested shapes, 
leading to some shapes being dereferenced multiple times. 
This is an implementation artifact rather than an inherent property of the method. 
Even in this worst case, the impact on the number of HTTP requests is small and does not affect the overall conclusions, 
especially given that requests are sent in parallel.}
	\label{tab:statSignificanceStateOfTheArt}
\end{table}